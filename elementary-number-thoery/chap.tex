%-*-latex-*-
\chapter{Elementary Number Theory}

The area of number theory huge. The scope of number theory requires it's own book and a alot of content. For this book, we will only talk about number theory as a study of $\Z$ and it's characmakteristic.
It is important to know that there is a lot more to number theory and Number htieory has given rise to different important topics in mathematics like  Automorphic thoery, theory of modular forms, algerbraic geometry etc.

Since, we are aware of the RSA cipher, we will study number theory up until the point of prime factorization as RSA is based on the difficulty of the prime factoerization of large prime numbers.

Let's start by talking about $\Z$ and it's operations.
\section{$\Z$}
We need to think of $Z$ as not just a single set but rather that it comes with the operation of addition $\add$ and multiplication $\mul$ as well as $0$ and $1$.
that is,
$( \Z, + , \mul , 1 ,0)$
Here are some properties of $\Z$:


\textbf{prime}\tinysidebar{prime}\index{prime}




\section{This is the second section}

42


\section{Einstein says}

$E = mc^2$
