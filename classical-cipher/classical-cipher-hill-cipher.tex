\sectionthree{Hill cipher}
\begin{python0}
from solutions import *; clear()
\end{python0}

\defone{Hill's cipher}
 Hill cipher, introduced by Lester S. Hill in 1929, represents an important advancement in cryptography by applying linear algebra to encryption. Rather than encrypting letters one at a time, it processes blocks of size $n$ via matrix multiplication.

\subsection*{Encryption Process}
For an $n\times n$ key matrix $K$ and plaintext vector $\mathbf{x}$ (converted A=0,\dots,Z=25),
\[
\mathbf{c} = K\mathbf{x} \bmod 26.
\]

A 3×3 example: $K = \begin{pmatrix}6 &24 &1 \\
13 &16 &10 \\
20 &17 &15\end{pmatrix}$, $\mathbf{x}=\begin{pmatrix}2 \\
17 \\
24\end{pmatrix}$ ("CRY").
\[
K\mathbf{x}=\begin{pmatrix}444 \\
538 \\
689\end{pmatrix}\equiv\begin{pmatrix}2 \\
18 \\
13\end{pmatrix}\bmod26
= \begin{pmatrix}C \\
S \\
N\end{pmatrix}.
\]

\subsection*{Decryption via Matrix Inversion}
Requires $K^{-1}$ in mod 26: compute $\det(K)$, adjugate, then multiply by $\det(K)^{-1}\bmod26$. Exists only if $\gcd(\det(K),26)=1$.

\subsection*{Key Requirements}
\begin{itemize}
  \item $K$ must be square and invertible mod 26.
  \item $\det(K)$ must be coprime with 26.
\end{itemize}

\subsection*{Small Example (2×2)}
Key $K=\begin{pmatrix}9 &4\\5 &7\end{pmatrix}$, "LINEAR"→ numeric blocks [11,8], [13,4], [0,17]. First block:
\[
K\begin{pmatrix}11\\8\end{pmatrix}=\begin{pmatrix}131\\111\end{pmatrix}\equiv\begin{pmatrix}1\\7\end{pmatrix}\bmod26=B,H.
\]
Results: "LINEAR"→"BHYCRX".

\subsection*{Advantages and Vulnerabilities}
\begin{itemize}
  \item Obscures single-letter frequencies via block processing.
  \item Vulnerable to known-plaintext attacks (solve for $K$ with pairs).
  \item Precursor to modern block ciphers.
\end{itemize}
