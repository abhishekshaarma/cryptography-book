\chapter{Axioms of $\Z$}
\begin{python0}
from solutions import *; clear()
\end{python0}
$(Z, +, \cdot, 0, 1)$ satisfies:

\textbf{Properties of $+$}
\begin{itemize}
\item \textbf{Closure}: $\forall x,y \in Z, x+y \in Z$
\item \textbf{Associativity}: $\forall x,y,z \in Z, (x+y)+z = x+(y+z)$
\item \textbf{Inverse}: $\forall x \in Z, \exists y$ s.t. $x+y=0=y+x$
\item \textbf{Neutrality}: $\forall x \in Z, 0+x=x=x+0$
\item \textbf{Commutativity}: $\forall x,y \in Z, x+y=y+x$
\end{itemize}

\textbf{Properties of $\cdot$}
\begin{itemize}
\item \textbf{Closure}: $\forall x,y \in Z, x \cdot y \in Z$
\item \textbf{Associativity}: $\forall x,y,z \in Z, (x \cdot y) \cdot z = x \cdot (y \cdot z)$
\item \textbf{Neutrality}: $\forall x \in Z, 1 \cdot x = x = x \cdot 1$
\item \textbf{Commutativity}: $\forall x,y \in Z, x \cdot y = y \cdot x$
\end{itemize}

\textbf{Distributivity}
$\forall x,y,z \in Z, x \cdot (y+z) = x \cdot y + x \cdot z$ and $(y+z) \cdot x = y \cdot x + z \cdot x$

\textbf{Ring Structure}\\
$R$ with ops $+_R, \cdot_R$ and elems $0_R, 1_R$ satisfying above = \textbf{commutative ring}.

Without commutativity = \textbf{non-commutative ring}.

Example: $M_{n \times n}(R)$ = non-commutative ring.

By convention, "ring" means commutative ring.

\textbf{Special Properties}
\begin{itemize}
\item \textbf{Integrality}: $\forall x,y \in Z, xy=0 \Rightarrow x=0 \text{ or } y=0$
\item \textbf{Nontriviality}: $0 \neq 1$
\end{itemize}

$Z$ is an \textbf{integral domain}.

\textbf{Peano-Dedekind Axioms for $\mathbb{N}$}
\begin{itemize}
\item \textbf{Induction}: If $X \subseteq \mathbb{N}$ with $0 \in X$ and $n \in X \Rightarrow n+1 \in X$, then $X = \mathbb{N}$
\end{itemize}

\textbf{Well-Ordering Principle}
\begin{itemize}
\item \textbf{WOP for $\mathbb{N}$}: If $X \subseteq \mathbb{N}$ non-empty, then $X$ has least element
\item \textbf{WOP for $Z$}: If $X \subseteq Z$ non-empty and bounded below, then $X$ has least element
\end{itemize}

\textbf{Induction Variants}

\textbf{For $\mathbb{N}$}
\begin{itemize}
\item \textbf{Weak Induction}: $0 \in X$ and $n \in X \Rightarrow n+1 \in X$ implies $X = \mathbb{N}$
\item \textbf{Strong Induction}: $0 \in X$ and $\forall k \leq n, k \in X \Rightarrow n+1 \in X$ implies $X = \mathbb{N}$
\end{itemize}

\textbf{For $Z$}
\begin{itemize}
\item \textbf{Weak Induction}: $P(n_0)$ true and $P(n) \Rightarrow P(n+1)$ implies $P(n)$ true $\forall n \geq n_0$
\item \textbf{Strong Induction}: $P(n_0)$ true and $[\forall k, n_0 \leq k \leq n, P(k)] \Rightarrow P(n+1)$ implies $P(n)$ true $\forall n \geq n_0$
\end{itemize}

\textbf{Order Axioms}
\begin{itemize}
\item \textbf{Trichotomy}: $\forall x \in Z$, exactly one: $-x \in Z^+$, $x = 0$, or $x \in Z^+$
\item \textbf{Closure of $+$ for $Z^+$}: $\forall x,y \in Z^+, x+y \in Z^+$
\item \textbf{Closure of $\cdot$ for $Z^+$}: $\forall x,y \in Z^+, x \cdot y \in Z^+$
\end{itemize}

Define $x < y$ if $y - x \in Z^+$

Define $x \leq y$ if $x < y$ or $x = y$

\textbf{Topology for $Z$}: $\forall x \in Z$, $\nexists y \in Z$ s.t. $x < y < x+1$

\textbf{Properties and Theorems}

\textbf{Prop 2.1.1}: Uniqueness of additive inverse.\\
If $x+y=0=y+x$ and $x+y'=0=y'+x$, then $y=y'$.

\textbf{Proof}:
$y = 0+y = (y'+x)+y = y'+(x+y) = y'+0 = y'$

\textbf{Def 2.1.1}: $x-y = x+(-y)$

\textbf{Def 2.1.2}: $y$ is multiplicative inverse of $x$ if $xy=1=yx$\\
$x$ is a unit if it has multiplicative inverse.

\textbf{Prop 2.1.2}: Uniqueness of multiplicative inverse.\\
If $xy=1=yx$ and $xy'=1=y'x$, then $y=y'$.

\textbf{Proof}:
$y = 1y = (y'x)y = y'(xy) = y'1 = y'$

\textbf{Def 2.1.3}: Mult. inverse is $x^{-1}$. Units: $U(Z)=Z^{\times}=\{-1,1\}$.

\textbf{Prop 2.1.3}: Cancellation law for addition.\\
(a) If $x+z=y+z$, then $x=y$.\\
(b) If $z+x=z+y$, then $x=y$.

\textbf{Prop 2.1.4}: Let $x \in Z$.\\
(a) $0x=0=x0$\\
(b) $-0=0$\\
(c) $x-0=x$

\textbf{Proof}:\\
(a) $0x=(0+0)x=0x+0x \Rightarrow 0+0x=0x+0x \Rightarrow 0=0x$\\
$0=0x=x0$ (by commutativity)\\
(b) $0+(-0)=0=(-0)+0$ and $0+0=0=0+0 \Rightarrow -0=0$\\
(c) $x-0=x+(-0)=x+0=x$

\textbf{Prop 2.1.5}: Let $x,y,c \in Z$.\\
(a) $-(-1)=1$\\
(b) $-(-x)=x$\\
(c) $x(-1)=-x=(-1)x$\\
(d) $(-1)(-1)=1$\\
(e) $(-x)(-y)=xy$\\
(f) $-(x+y)=-x+-y$\\
(g) $-(x-y)=-x+y$

\textbf{Proof}:\\
(b) $(-x)+(-(-x))=0=(-(-x))+(-x)$ and $(-x)+x=0=x+(-x) \Rightarrow -(-x)=x$\\
(a) From (b) with $x=1$, $-(-1)=1$\\
(c) $x+x(-1)=x \cdot 1+x(-1)=x(1+(-1))=x0=0 \Rightarrow x(-1)=-x$\\
(d) $(-1)(-1)=-(-1)=1$\\
(e) $(-x)(-y)=(-1)x(-1)y=(-1)(-1)xy=1xy=xy$\\
(f) $-(x+y)=(-1)(x+y)=(-1)x+(-1)y=-x+-y$\\
(g) $-(x-y)=-(x+(-y))=(-1)(x+(-y))=(-1)x+(-1)(-y)=-x+(-(-y))=-x+y$

\textbf{Prop 2.1.6}: Cancellation law for multiplication.\\
(a) If $xz=yz$ and $z \neq 0$, then $x=y$.\\
(b) If $zx=zy$ and $z \neq 0$, then $x=y$.

\textbf{Proof}:\\
$xz=yz \Rightarrow xz+(-yz)=0 \Rightarrow (x+(-1)y)z=0 \Rightarrow x+(-1)y=0$ or $z=0$\\
Since $z \neq 0$, $x+(-1)y=0 \Rightarrow x=-(-1)y=(-1)(-1)y=1y=y$

\textbf{Formal Sums and Products:}
$\sum_{i=1}^n x_i = \begin{cases}
0 & \text{if } n=0 \\
\sum_{i=1}^{n-1} x_i + x_n & \text{if } n > 0
\end{cases}$

$\prod_{i=1}^n x_i = \begin{cases}
1 & \text{if } n=0 \\
\prod_{i=1}^{n-1} x_i \cdot x_n & \text{if } n > 0
\end{cases}$
