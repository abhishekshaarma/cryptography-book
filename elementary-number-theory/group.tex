\chapter{Semi-Groups}
\begin{python0}
  from solutions import *; clear()
\end{python0}


Let us recall that a group is$(G, *, e)$ where $G$ is a set and $e \in G$ such that it satisfies

\begin{enumerate}
\item \textbf{Closure:}  
  If $x, y \in G$, then $x * y \in G$.  
  This means that $* : G \times G \to G$ is a binary operation.
\item \textbf{Associativity:}  
  If $x, y, z \in G$, then $(x * y) * z = x * ( y * z )$.  
\item \textbf{Inverse:}  
  If $x \in  G$, then there is some $y \in G$ such that  $x * y = e = y * x$.  

\item \textbf{Neutral:}  
  If $x \in  G$, then   $e * x = x = x * e$.  


\end{enumerate}

For a semigroup, it is almost a group except you do not need the inverses.


\textbf{Definition:}
A semigroup is a tuple $(G, *)$ where $G$ where $G$ is a set and the following are satisfied:

\begin{enumerate}
\item \textbf{Closure:}  
  If $x, y \in G$, then $x * y \in G$.  
  This means that $* : G \times G \to G$ is a binary operation.
\item \textbf{Associativity:}  
  If $x, y, z \in G$, then $(x * y) * z = x * ( y * z )$.  

\end{enumerate}

Commulative Semigroup $(G, *)$ is a semigroup such that $*$ is commulative, i.e,
if $x , y \in G$, then

\textbf{Definition:}
A monoid is a tuple $(G, *, e)$ where G is a set and the following are satisfied:
Closure, Associativity and Neutral.

And of course a commulative monoid $(G, *, e)$ is a monoid such that $*$ is commulative, i.e, if $x, y \in G$, then
$x*y = y*x$


\textbf{Proposition:} Uniqueness of an Identity.

Suppose e, f are identities in $G$.


$\forall a \ in G$
$ae = ea = a$

and
$af = fa = a$
Let us take, $a  = f$, then,
$fe = ef = f$
now, let's assume $a = e$, then
$ef = fe = e$

The above statements are only true for $e = f$, thus proving uniqueness.

