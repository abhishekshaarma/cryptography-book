%-*-latex-*-
\chapter{Divisibility}
\begin{python0}
from solutions import *; clear()
\end{python0}

\textbf{Def 2.2.1}: Let $a, n \in Z$ with $a \neq 0$. We say $a$ divides $b$, written $a \mid b$, if $\exists x \in Z$ s.t. $ax = b$.

\textbf{Prop 2.2.1}: Let $a, b, c \in Z$.
\begin{itemize}
\item (a) $1 \mid a$.
\item (b) $a \mid 0$.
\item (c) Reflexivity: $a \mid a$.
\item (d) Transitivity: If $a \mid b$ and $b \mid c$, then $a \mid c$.
\item (e) Antisymmetry: If $a \mid b$ and $b \mid a$, then $a = \pm b$.
\item (f) If $a \mid b$, then $a \mid bc$.
\item (g) If $a \mid b$ and $a \mid c$, then $a \mid b + c$.
\item (h) Linearity: If $a \mid b, a \mid c$, then $a \mid bx + cy$ for $x, y \in Z$.
\item (i) If $a \mid b$, then $|a| \leq |b|$.
\end{itemize}

\textbf{Proof}:
\begin{itemize}
\item (a) $1 \cdot a = a \Rightarrow 1 \mid a$.
\item (b) $a \cdot 0 = 0 \Rightarrow a \mid 0$.
\item (c) $a \cdot 1 = a \Rightarrow a \mid a$.
\item (d) If $a \mid b, b \mid c$, then $\exists x,y \in Z$ s.t. $ax = b, by = c$. Thus $axy = c \Rightarrow a \mid c$.
\item (e) If $a \mid b, b \mid a$, then $\exists x,y \in Z$ s.t. $ax = b, by = a$. Thus $bxy = b$, so $b(xy - 1) = 0$. Since $b \neq 0$, $xy - 1 = 0 \Rightarrow xy = 1$. Hence $x = y = 1$ or $x = y = -1$, giving $a = b$ or $a = -b$.
\item (f) If $a \mid b$, then $ax = b$. Thus $axc = bc \Rightarrow a \mid bc$.
\item (g) If $a \mid b, a \mid c$, then $ax = b, ay = c$. Thus $a(x + y) = ax + ay = b + c \Rightarrow a \mid b + c$.
\item (h) If $a \mid b, a \mid c$, then by (f), $a \mid bx, a \mid cy$. By (g), $a \mid bx + cy$.
\item (i) If $a \mid b$, then $ax = b$ for some $x \in Z$. Thus $|a||x| = |ax| = |b| \Rightarrow |a| \leq |b|$.
\end{itemize}

\textbf{Congruences}

\textbf{Def 2.3.1}: Let $a, b \in Z$ and $N \in Z$ with $N > 0$. Then $a$ is congruent to $b$ mod $N$, written $a \equiv b \pmod{N}$, if $N \mid a-b$.

\textbf{Prop 2.3.1}: Let $a, b, c, a', b' \in Z$ and $N, N' \geq 0$ be in $Z$.
\begin{itemize}
\item (a) Reflexivity: $a \equiv a \pmod{N}$
\item (b) Symmetry: If $a \equiv b \pmod{N}$, then $b \equiv a \pmod{N}$
\item (c) Transitivity: If $a \equiv b, b \equiv c \pmod{N}$, then $a \equiv c \pmod{N}$
\item (d) Additivity: If $a \equiv b, a' \equiv b' \pmod{N}$, then $a + a' \equiv b + b' \pmod{N}$
\item (e) Multiplicativity: If $a \equiv b, a' \equiv b' \pmod{N}$, then $aa' \equiv bb' \pmod{N}$
\item (f) If $a \equiv b \pmod{NN'}$, then $a \equiv b \pmod{N}$
\end{itemize}

\textbf{Prop 2.3.2}: Let $a, N \in Z$ with $N > 0$. Let $q, r \in Z$ such that $a = Nq + r, 0 \leq r < N$. Then $a \equiv r \pmod{N}$.

\textbf{Def 2.3.2}: Let $a, N \in Z$ with $N > 0$. By Euclidean property of $Z$, $\exists$ unique $q, r$ s.t. $a = Nq + r, 0 \leq r < N$. $r$ is called "residue of $a$ mod $N$" (remainder after division). Written as $a \bmod N$ or $r_N(a)$.

Example: For $15 \bmod 4$, $15 = 4 \cdot 3 + 3$ where $0 \leq 3 < 4$. So $15 \equiv 3 \pmod{4}$ and residue $r_4(15) = 3$.

Warning: "mod" has two meanings:
\begin{itemize}
\item Relation: $a \equiv b \pmod{N}$
\item Function: $a \bmod N = r$
\end{itemize}
