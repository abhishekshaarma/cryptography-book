\chapter{Rings and Fields}
\begin{python0}
  from solutions import *; clear()
\end{python0}

We can generalize the properties of $\Z$ using rings.

\textbf Definition: $(R,+_{R}, *_{R}, 0_R, 1_R)$ is a ring if
\begin{enumerate}
  \item $(R, +_{R} 0_R)$ is an abelian group
  \item $(R, ._{R} 1_R)$ is an semigroup with Identity
  \item Distribution: If $ x, y, z \in R$, then,
    $x *_R (y +_R z) = x *_R y +_R x *_r z$
    $ (y +_R z)*_R x  = y *_R x +_R z *_r x$

\end{enumerate}

$R,+_{R}, *_{R}, 0_R, 1_R)$  is a commulative ring if it is a ring and if $x, y \in R$,

$x *_R y = y *_R x$

So, A ring R is a set of stuff with two operation that. and they also have two different things known as the additive identity and the multiplicative identity.
One way to easily visualize this is by thinking of integers, We have addition and multiplication as two operators and then for additive identity, we have $0$, no matter what element is added to $0$, the answer is always the element. Same holds true for multiplication and $1$.

